% Options for packages loaded elsewhere
\PassOptionsToPackage{unicode}{hyperref}
\PassOptionsToPackage{hyphens}{url}
%
\documentclass[
]{article}
\usepackage{amsmath,amssymb}
\usepackage{lmodern}
\usepackage{iftex}
\ifPDFTeX
  \usepackage[T1]{fontenc}
  \usepackage[utf8]{inputenc}
  \usepackage{textcomp} % provide euro and other symbols
\else % if luatex or xetex
  \usepackage{unicode-math}
  \defaultfontfeatures{Scale=MatchLowercase}
  \defaultfontfeatures[\rmfamily]{Ligatures=TeX,Scale=1}
\fi
% Use upquote if available, for straight quotes in verbatim environments
\IfFileExists{upquote.sty}{\usepackage{upquote}}{}
\IfFileExists{microtype.sty}{% use microtype if available
  \usepackage[]{microtype}
  \UseMicrotypeSet[protrusion]{basicmath} % disable protrusion for tt fonts
}{}
\makeatletter
\@ifundefined{KOMAClassName}{% if non-KOMA class
  \IfFileExists{parskip.sty}{%
    \usepackage{parskip}
  }{% else
    \setlength{\parindent}{0pt}
    \setlength{\parskip}{6pt plus 2pt minus 1pt}}
}{% if KOMA class
  \KOMAoptions{parskip=half}}
\makeatother
\usepackage{xcolor}
\IfFileExists{xurl.sty}{\usepackage{xurl}}{} % add URL line breaks if available
\IfFileExists{bookmark.sty}{\usepackage{bookmark}}{\usepackage{hyperref}}
\hypersetup{
  pdftitle={Covid-19 isolation and quarantine orders in a district of Berlin, Germany How many, how long, to whom and predictive factors},
  pdfauthor={Jakob Schumacher, Lisa Kühne, Sophie Bruessermann, Benjamin Geisler, Sonja Jäckle},
  hidelinks,
  pdfcreator={LaTeX via pandoc}}
\urlstyle{same} % disable monospaced font for URLs
\usepackage[margin=1in]{geometry}
\usepackage{color}
\usepackage{fancyvrb}
\newcommand{\VerbBar}{|}
\newcommand{\VERB}{\Verb[commandchars=\\\{\}]}
\DefineVerbatimEnvironment{Highlighting}{Verbatim}{commandchars=\\\{\}}
% Add ',fontsize=\small' for more characters per line
\usepackage{framed}
\definecolor{shadecolor}{RGB}{248,248,248}
\newenvironment{Shaded}{\begin{snugshade}}{\end{snugshade}}
\newcommand{\AlertTok}[1]{\textcolor[rgb]{0.94,0.16,0.16}{#1}}
\newcommand{\AnnotationTok}[1]{\textcolor[rgb]{0.56,0.35,0.01}{\textbf{\textit{#1}}}}
\newcommand{\AttributeTok}[1]{\textcolor[rgb]{0.77,0.63,0.00}{#1}}
\newcommand{\BaseNTok}[1]{\textcolor[rgb]{0.00,0.00,0.81}{#1}}
\newcommand{\BuiltInTok}[1]{#1}
\newcommand{\CharTok}[1]{\textcolor[rgb]{0.31,0.60,0.02}{#1}}
\newcommand{\CommentTok}[1]{\textcolor[rgb]{0.56,0.35,0.01}{\textit{#1}}}
\newcommand{\CommentVarTok}[1]{\textcolor[rgb]{0.56,0.35,0.01}{\textbf{\textit{#1}}}}
\newcommand{\ConstantTok}[1]{\textcolor[rgb]{0.00,0.00,0.00}{#1}}
\newcommand{\ControlFlowTok}[1]{\textcolor[rgb]{0.13,0.29,0.53}{\textbf{#1}}}
\newcommand{\DataTypeTok}[1]{\textcolor[rgb]{0.13,0.29,0.53}{#1}}
\newcommand{\DecValTok}[1]{\textcolor[rgb]{0.00,0.00,0.81}{#1}}
\newcommand{\DocumentationTok}[1]{\textcolor[rgb]{0.56,0.35,0.01}{\textbf{\textit{#1}}}}
\newcommand{\ErrorTok}[1]{\textcolor[rgb]{0.64,0.00,0.00}{\textbf{#1}}}
\newcommand{\ExtensionTok}[1]{#1}
\newcommand{\FloatTok}[1]{\textcolor[rgb]{0.00,0.00,0.81}{#1}}
\newcommand{\FunctionTok}[1]{\textcolor[rgb]{0.00,0.00,0.00}{#1}}
\newcommand{\ImportTok}[1]{#1}
\newcommand{\InformationTok}[1]{\textcolor[rgb]{0.56,0.35,0.01}{\textbf{\textit{#1}}}}
\newcommand{\KeywordTok}[1]{\textcolor[rgb]{0.13,0.29,0.53}{\textbf{#1}}}
\newcommand{\NormalTok}[1]{#1}
\newcommand{\OperatorTok}[1]{\textcolor[rgb]{0.81,0.36,0.00}{\textbf{#1}}}
\newcommand{\OtherTok}[1]{\textcolor[rgb]{0.56,0.35,0.01}{#1}}
\newcommand{\PreprocessorTok}[1]{\textcolor[rgb]{0.56,0.35,0.01}{\textit{#1}}}
\newcommand{\RegionMarkerTok}[1]{#1}
\newcommand{\SpecialCharTok}[1]{\textcolor[rgb]{0.00,0.00,0.00}{#1}}
\newcommand{\SpecialStringTok}[1]{\textcolor[rgb]{0.31,0.60,0.02}{#1}}
\newcommand{\StringTok}[1]{\textcolor[rgb]{0.31,0.60,0.02}{#1}}
\newcommand{\VariableTok}[1]{\textcolor[rgb]{0.00,0.00,0.00}{#1}}
\newcommand{\VerbatimStringTok}[1]{\textcolor[rgb]{0.31,0.60,0.02}{#1}}
\newcommand{\WarningTok}[1]{\textcolor[rgb]{0.56,0.35,0.01}{\textbf{\textit{#1}}}}
\usepackage{graphicx}
\makeatletter
\def\maxwidth{\ifdim\Gin@nat@width>\linewidth\linewidth\else\Gin@nat@width\fi}
\def\maxheight{\ifdim\Gin@nat@height>\textheight\textheight\else\Gin@nat@height\fi}
\makeatother
% Scale images if necessary, so that they will not overflow the page
% margins by default, and it is still possible to overwrite the defaults
% using explicit options in \includegraphics[width, height, ...]{}
\setkeys{Gin}{width=\maxwidth,height=\maxheight,keepaspectratio}
% Set default figure placement to htbp
\makeatletter
\def\fps@figure{htbp}
\makeatother
\setlength{\emergencystretch}{3em} % prevent overfull lines
\providecommand{\tightlist}{%
  \setlength{\itemsep}{0pt}\setlength{\parskip}{0pt}}
\setcounter{secnumdepth}{-\maxdimen} % remove section numbering
\ifLuaTeX
  \usepackage{selnolig}  % disable illegal ligatures
\fi

\title{Covid-19 isolation and quarantine orders in a district of Berlin,
Germany How many, how long, to whom and predictive factors}
\author{Jakob Schumacher, Lisa Kühne, Sophie Bruessermann, Benjamin
Geisler, Sonja Jäckle}
\date{06. Mai 2022}

\begin{document}
\maketitle

\begin{Shaded}
\begin{Highlighting}[]
\CommentTok{\# ![]("https://www.horizont.net/news/media/32/Das{-}neue{-}Berlin{-}Logo{-}315209.jpeg")\{width=30\%\}}
\CommentTok{\# ![]("https://www.ultrasoundsymposium.org/wp{-}content/uploads/2017/08/fhg{-}1.gif")\{width=30\%\}}
\CommentTok{\# ![](https://www.lsc{-}digital{-}public{-}health.de/images/partners/leibniz{-}institut{-}bips.png)\{width=30\%\}}


\DocumentationTok{\#\#\#\#\#\#\#\#\#\#\#\#\#\#\#\#\#\#\#\#\#\#\#\#\#\#\#\#\#\#\#\#\#\#\#\#\#\#\#\#\#\#\#\#\#\#\#}
\CommentTok{\# Setting options for knitr, ggplot, fonts }
\DocumentationTok{\#\#\#\#\#\#\#\#\#\#\#\#\#\#\#\#\#\#\#\#\#\#\#\#\#\#\#\#\#\#\#\#\#\#\#\#\#\#\#\#\#\#\#\#\#\#}

\CommentTok{\# Disabeling scientific notation}
\FunctionTok{options}\NormalTok{(}\AttributeTok{scipen =} \DecValTok{999}\NormalTok{)}

\CommentTok{\# Create correct figure caption}
\NormalTok{knitr}\SpecialCharTok{::}\NormalTok{opts\_knit}\SpecialCharTok{$}\FunctionTok{set}\NormalTok{(}\AttributeTok{eval.after =} \StringTok{\textquotesingle{}fig.cap\textquotesingle{}}\NormalTok{)}

\CommentTok{\# Adjust the big mark for large numbers}
\NormalTok{knitr}\SpecialCharTok{::}\NormalTok{knit\_hooks}\SpecialCharTok{$}\FunctionTok{set}\NormalTok{(}\AttributeTok{inline =} \ControlFlowTok{function}\NormalTok{(x) \{ }\FunctionTok{prettyNum}\NormalTok{(x, }\AttributeTok{big.mark=}\StringTok{" "}\NormalTok{) \})}
\end{Highlighting}
\end{Shaded}

\hypertarget{about-this-repository}{%
\section{About this Repository}\label{about-this-repository}}

The following R-Script calculates all the necessary numbers and figures
for a publication. All necessary files to reproduce are available. The
analysis is done in R. This project uses Renv. See the file .Rprofile
for used packages. This script runs with the package target. The
important parts of the script lie in the functions in the code folder.
You can check the file \_targets.R to see the different steps in their
sequential order.

\hypertarget{results}{%
\section{Results}\label{results}}

\begin{Shaded}
\begin{Highlighting}[]
\NormalTok{df }\OtherTok{\textless{}{-}} \FunctionTok{tar\_read}\NormalTok{(df)}
\NormalTok{demographiedaten }\OtherTok{\textless{}{-}} \FunctionTok{tar\_read}\NormalTok{(demographiedaten)}
\NormalTok{zeiten }\OtherTok{\textless{}{-}} \FunctionTok{tar\_read}\NormalTok{(externalinput)}\SpecialCharTok{$}\NormalTok{zeiten}
\NormalTok{externalinput }\OtherTok{\textless{}{-}} \FunctionTok{tar\_read}\NormalTok{(externalinput)}
\NormalTok{resultslist }\OtherTok{\textless{}{-}} \FunctionTok{tar\_read}\NormalTok{(results)}
\end{Highlighting}
\end{Shaded}

We extracted 109 087 datasets from SurvNet. 73 220 entries fullfilled
the definition (11 215 had missing dates, 108 entries had an IDs that
did not lead to an existing person and 24 563 separation orders did not
begin in the study period). We removed 371 entries because they had a
presumed typing error in one of the dates. We also removed 30 duplicated
isolations and 2 497 duplicated quarantines. For 3 484 quarantines we
reduced the length by the overlap with a following isolation period. In
the demographic data we found 266 123 inhabitants registered in Berlin
Reinickendorf (18 084 in the age group \textless7; 27 001 in the age
group 7-17; 158 199 in the age group 18-64 and 62 839 in the age group
\textgreater64).

\hypertarget{results-of-statistical-measures}{%
\subsection{Results of statistical
measures}\label{results-of-statistical-measures}}

\begin{itemize}
\item
  \emph{Analysis of quantity of isolation and quarantines:} The local
  public health agency ordered \(n_{\text{i}}\) = 24 433 isolations and
  \(n_{\text{q}}\) = 45 335 quarantines (\(n_{\text{i-p100}}\) = 9.2
  isolations and \(n_{\text{q-p100}}\) = 17 quarantines per 100
  inhabitants). The number of quarantines and isolations by age group
  and recommendation period can bee seen in @ref(tab:agegrouptable)).
  Remarkably the number of quarantines per 100 inhabitants
  \(n_{\text{q-p100}}\) was resultslist\(agegroup_table\)q\_p{[}1{]} for
  the age group of children aged 0 to 6 and
  resultslist\(agegroup_table\)q\_p{[}2{]} in the age group 7 to 17
  compared to resultslist\(agegroup_table\)q\_p{[}3{]} in adults or
  resultslist\(agegroup_table\)q\_p{[}4{]} in the elderly. 46 817 (81.5
  \%) of persons had one spearation order (quarantine or isolation), 9
  061 (15.8 \%) had two spearation orders, 1 359 (2.4 \%) had three
  spearation orders, 163 (0.3 \%) had four spearation orders and 20 had
  five spearation orders - the maximum.
\item
  \emph{Analysis of the duration of isolation quarantines:} The median
  duration for isolations was \(\tilde d_{\text{i}}\) = 10 days
  (interquartile range 8 - 13). The duration did change in between
  different periods of recommendations. The median of the duration
  during the recommendation periods were: 14 days for the period No.~1,
  8 days for the period No.~2 and 12 days for the period No.~3. The
  overall median duration for quarantines was \(\tilde d_{\text{q}}\) =
  8 days (interquartile range 6 - 11). The median duration did differ
  between periods of different recommendations and age groups. The
  median of the duration during the recommendation periods were: 9 days
  for the period No.~1, 9 days for the period No.~2, 10 days for the
  period No.~3 and 4 days for the period No.~4. See Fig
  @ref(fig:duration). All together the public health agency ordered 684
  years of isolations and 1 031 years of quarantine or 1 714 years in
  total.
\item
  \emph{Analysis of the ratio of contact persons per case:} The overal
  ratio of contact persons was \(r_{\text{qi}}\) = 1.89. In the period
  of the contact person defintion no. 1 the ratio was 2.88 in the period
  no. 2 the ratio was 1.96 and in the period no. 3 the ratio was: 0.95.
\item
  \emph{Analysis of isolations following quarantines:} In the time
  period from the start of the recording of quarantines 3 483 of 23 892
  isolations had a directly preceeding quarantine and 532 a preceeding
  quarantine in the 1 to 7 days before the isolations. 3 483 of 45 272
  quarantines in that time period had a directly following isolation
  (contained case) and 535 a isolation following in the days 1 to 7
  after the quarantine (non-contained case). This did differ between
  different periods and recommendations see Fig
  @ref(fig:adjoining-quarantines-and-isolation).
\item
  \emph{Reduction of the reproductive number:} Assuming a total
  prevention of transmission by the quarantine order we calculated a
  reduction of 0.15 of the reproductive Number by quarantine orders.
\item
  \emph{Analysis of timeliness:} Our approximation of the median time
  period between the last contact and the beginning of the quarantine
  order was \(\tilde d_{\text{delay}}\) = 4 (interquartile range 1 - 6)
  during the time periods when 14 days were recommended as a quarantine
  duration.
\end{itemize}

\hypertarget{all-results}{%
\section{All results}\label{all-results}}

\begin{Shaded}
\begin{Highlighting}[]
\NormalTok{resultslist}
\end{Highlighting}
\end{Shaded}

\begin{verbatim}
## $queried
## [1] 109087
## 
## $emptydates
## [1] 11215
## 
## $wrongid
## [1] 108
## 
## $outofrange
## [1] 24563
## 
## $definitionfullfilled
## [1] 73220
## 
## $typingerror
## [1] 371
## 
## $deleted_duplicates_table
##  DatensatzKategorie deleted_duplicates
##            COVID-19                 30
##    Kontakt-COVID-19               2497
## 
## $deleted_duplicates_quarantines
## [1] 2497
## 
## $deleted_duplicates_isolations
## [1] 30
## 
## $adjustedQuarantines
## [1] 3484
## 
## $N
## [1] 266123
## 
## $N_0_6
## [1] 18084
## 
## $N_7_17
## [1] 27001
## 
## $N_18_64
## [1] 158199
## 
## $N_65_110
## [1] 62839
## 
## $I_n
## [1] 24433
## 
## $Q_n
## [1] 45335
## 
## $I_p
## [1] 9.2
## 
## $Q_p
## [1] 17
## 
## $totaltime_groups
## # A tibble: 8 x 7
##   DatensatzKategor~ AgeGroup completeduratio~ completeduratio~  value percentage
##   <chr>             <chr>               <dbl>            <dbl>  <dbl>      <dbl>
## 1 COVID-19          0 to 6              15491             42.4  18084        6.2
## 2 COVID-19          7 to 17             44356            122.   27001       17.8
## 3 COVID-19          18 to 64           161215            442.  158199       64.6
## 4 COVID-19          65 to 1~            28445             77.9  62839       11.4
## 5 Kontakt-COVID-19  0 to 6              74977            205.   18084       19.9
## 6 Kontakt-COVID-19  7 to 17            139069            381.   27001       37  
## 7 Kontakt-COVID-19  18 to 64           145456            399.  158199       38.7
## 8 Kontakt-COVID-19  65 to 1~            16712             45.8  62839        4.4
## # ... with 1 more variable: completeduration_person <dbl>
## 
## $QundIproPerson_table
## # A tibble: 5 x 3
##   number     n     p
##    <int> <int> <dbl>
## 1      1 46817  81.5
## 2      2  9061  15.8
## 3      3  1359   2.4
## 4      4   163   0.3
## 5      5    20   0  
## 
## $QundIproPerson_1_order_n
## [1] 46817
## 
## $QundIproPerson_1_order_p
## [1] 81.5
## 
## $QundIproPerson_2_order_n
## [1] 9061
## 
## $QundIproPerson_2_order_p
## [1] 15.8
## 
## $QundIproPerson_3_order_n
## [1] 1359
## 
## $QundIproPerson_3_order_p
## [1] 2.4
## 
## $QundIproPerson_4_order_n
## [1] 163
## 
## $QundIproPerson_4_order_p
## [1] 0.3
## 
## $QundIproPerson_5_order_n
## [1] 20
## 
## $QundIproPerson_5_order_p
## [1] 0
## 
## $MedianeDauerI
##   0%  25%  50%  75% 100% 
##    1    8   10   13   30 
## 
## $MedianeDauerI_Rec
## # A tibble: 3 x 2
##   I_Duration   quint
##   <chr>        <dbl>
## 1 I_Duration_1    14
## 2 I_Duration_2     8
## 3 I_Duration_3    12
## 
## $MedianeDauerI_Rec_1
## 50% 
##  14 
## 
## $MedianeDauerI_Rec_2
## 50% 
##   8 
## 
## $MedianeDauerI_Rec_3
## 50% 
##  12 
## 
## $MedianeDauerQ
##   0%  25%  50%  75% 100% 
##    1    6    8   11   28 
## 
## $MedianeDauerQ_Rec
## # A tibble: 4 x 2
##   Q_Duration   quint
##   <chr>        <dbl>
## 1 Q_Duration_1     9
## 2 Q_Duration_2     9
## 3 Q_Duration_3    10
## 4 Q_Duration_4     4
## 
## $MedianeDauerQ_Rec_1
## 50% 
##   9 
## 
## $MedianeDauerQ_Rec_2
## 50% 
##   9 
## 
## $MedianeDauerQ_Rec_3
## 50% 
##  10 
## 
## $MedianeDauerQ_Rec_4
## 50% 
##   4 
## 
## $qi_d
## [1] 625721
## 
## $qi_d_in_y
## [1] 1714
## 
## $q_d_in_y
## [1] 1031
## 
## $i_d_in_y
## [1] 684
## 
## $K_F_Verhaeltnis
## [1] 1.89
## 
## $K_F_Verhaeltnis_QDef
## # A tibble: 3 x 4
##   q_def   covid_19 kontakt_covid_19 verhaeltnis
##   <chr>      <int>            <int>       <dbl>
## 1 Q_Def_1    10402            29965        2.88
## 2 Q_Def_2     2446             4791        1.96
## 3 Q_Def_3    11044            10516        0.95
## 
## $K_F_Verhaeltnis_QDef_1
## [1] 2.88
## 
## $K_F_Verhaeltnis_QDef_2
## [1] 1.96
## 
## $K_F_Verhaeltnis_QDef_3
## [1] 0.95
## 
## $I_after_Q
## $I_after_Q$I_correct_after_Q
## [1] 3483
## 
## $I_after_Q$I_too_long_after_Q
## [1] 532
## 
## $I_after_Q$No_I_after_Q
## [1] 19877
## 
## 
## $I_n_kptime
## [1] 23892
## 
## $Q_with_I_after
## $Q_with_I_after$I_correct_after_Q
## [1] 3483
## 
## $Q_with_I_after$I_too_long_after_Q
## [1] 535
## 
## $Q_with_I_after$No_I_after_Q
## [1] 41254
## 
## 
## $Q_n_kptime
## [1] 45272
## 
## $r
## [1] 0.15
## 
## $Q_n_by_QDef
## # A tibble: 3 x 2
##   Q_Def       n
##   <chr>   <int>
## 1 Q_Def_1 29965
## 2 Q_Def_2  4791
## 3 Q_Def_3 10516
## 
## $Q_with_correct_I_by_QDef_table
## # A tibble: 3 x 5
## # Groups:   Q_Def [3]
##   Q_Def   result                n     N percentage
##   <chr>   <chr>             <int> <int>      <dbl>
## 1 Q_Def_1 I_correct_after_Q  1802 29965          6
## 2 Q_Def_2 I_correct_after_Q   658  4791         14
## 3 Q_Def_3 I_correct_after_Q  1024 10516         10
## 
## $Q_with_too_late_I_by_QDef_table
## # A tibble: 3 x 5
## # Groups:   Q_Def [3]
##   Q_Def   result                 n     N percentage
##   <chr>   <chr>              <int> <int>      <dbl>
## 1 Q_Def_1 I_too_long_after_Q   205 29965        0.7
## 2 Q_Def_2 I_too_long_after_Q    52  4791        1.1
## 3 Q_Def_3 I_too_long_after_Q   278 10516        2.6
## 
## $Q_n_by_AgeGroup
## # A tibble: 4 x 2
##   AgeGroup      n
##   <ord>     <int>
## 1 0 to 6     9149
## 2 7 to 17   17528
## 3 18 to 64  16678
## 4 65 to 110  1980
## 
## $Q_with_correct_I_by_Agegroup_table
## # A tibble: 4 x 5
## # Groups:   AgeGroup [4]
##   AgeGroup  result                n     N percentage
##   <ord>     <chr>             <int> <int>      <dbl>
## 1 0 to 6    I_correct_after_Q   434  9149        4.7
## 2 7 to 17   I_correct_after_Q   867 17528        4.9
## 3 18 to 64  I_correct_after_Q  1838 16678       11  
## 4 65 to 110 I_correct_after_Q   345  1980       17.4
## 
## $Q_with_too_late_I_by_Agegroup_table
## # A tibble: 4 x 5
## # Groups:   AgeGroup [4]
##   AgeGroup  result                 n     N percentage
##   <ord>     <chr>              <int> <int>      <dbl>
## 1 0 to 6    I_too_long_after_Q    97  9149        1.1
## 2 7 to 17   I_too_long_after_Q   194 17528        1.1
## 3 18 to 64  I_too_long_after_Q   210 16678        1.3
## 4 65 to 110 I_too_long_after_Q    34  1980        1.7
## 
## $q_timeliness_median
##   0%  25%  50%  75% 100% 
##    0    1    4    6   12 
## 
## $agegroup_table
## # A tibble: 4 x 16
##   AgeGroup       N   q_n   i_n   q_p   i_p   q_d   i_d q_sum_in_y i_sum_in_y
##   <chr>      <dbl> <int> <int> <dbl> <dbl> <dbl> <dbl>      <dbl>      <dbl>
## 1 0 to 6     18084  9149  1383  50.6   7.6   8.2  11.2      205.        42.4
## 2 7 to 17    27001 17528  3983  64.9  14.8   7.9  11.1      381        122. 
## 3 18 to 64  158199 16678 16041  10.5  10.1   8.7  10.1      398.       442. 
## 4 65 to 110  62839  1980  3026   3.2   4.8   8.4   9.4       45.8       77.9
## # ... with 6 more variables: q_sum_in_d_per_p <dbl>, i_sum_in_d_per_p <dbl>,
## #   contained <int>, containedp <dbl>, toolate <int>, toolatep <dbl>
## 
## $qdef_table
## # A tibble: 3 x 16
##   Q_Def        N   q_n   i_n   q_p   i_p   q_d   i_d q_sum_in_y i_sum_in_y
##   <chr>    <dbl> <int> <int> <dbl> <dbl> <dbl> <dbl>      <dbl>      <dbl>
## 1 Q_Def_1 266123 29973 10876  11.3   4.1   8.9   8.3       734.      248. 
## 2 Q_Def_2 266123  4791  2446   1.8   0.9   9.5  11.4       124.       76.2
## 3 Q_Def_3 266123 10571 11111   4     4.2   5.9  11.8       172.      360. 
## # ... with 6 more variables: q_sum_in_d_per_p <dbl>, i_sum_in_d_per_p <dbl>,
## #   contained <int>, containedp <dbl>, toolate <int>, toolatep <dbl>
## 
## $total_table
## # A tibble: 1 x 16
##   total      N   q_n   i_n   q_p   i_p   q_d   i_d q_sum_in_y i_sum_in_y
##   <chr>  <dbl> <int> <int> <dbl> <dbl> <dbl> <dbl>      <dbl>      <dbl>
## 1 total 266123 45335 24433    17   9.2   8.3  10.2      1031.       684.
## # ... with 6 more variables: q_sum_in_d_per_p <dbl>, i_sum_in_d_per_p <dbl>,
## #   contained <int>, containedp <dbl>, toolate <int>, toolatep <dbl>
\end{verbatim}

\end{document}
